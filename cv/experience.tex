%-------------------------------------------------------------------------------
%	SECTION TITLE
%-------------------------------------------------------------------------------
\cvsection{RESEARCH EXPERIENCE}

%-------------------------------------------------------------------------------
%	CONTENT
%-------------------------------------------------------------------------------
\begin{cventries2}


%---------------------------------------------------------
  \cventrynew
    {A Domain Knowledge Embedding Framework of Operator Learning and Brain-Inspired Computing} % Job title
    {07/2022-Current} % Date (s)
    {\textnormal{\textit{Supervised by Prof. Peng Hao of Department of Engineering Mechanics of DLUT}
    }}
    {
      \begin{cvitems} % Description(s) of tasks/responsibilities
        % \item {Developed and deployed {\bf RL} pipeline for robotic systems, integrating {\bf physics engines} ({\bf ROS}) to solve real-world problems.} 
        % \item {Drove advanced solutions, optimizing robotic platform performance under uncertainty through {\bf AI-driven systems}.}
        \item Proposed variational operator learning (VOL), a unified paradigm for learning neural operators and solving partial differential equations; introduced Ritz and Galerkin approaches specifically tailored for the VOL framework, achieving the approximation of energy functional and calculation of residual in a matrix-free manner, with linear time complexity and constant ($O(1)$) space complexity
		    \item Introduced direct minimization and iterative update as two optimization strategies into the framework of VOL to minimize the residual norm; integrated the steepest decent method (SD) and conjugate gradient method (CG) into VOL with an efficient restart-update manner for iterative update strategy
		    \item Investigated VOL with various experimental results; conducted scaling experiments, resolution experiments, comparative experiments verifying generalization benefits of VOL, comparative experiments for different optimization strategies, and comparative experiments between VOL and existing physics-informed operator learning approaches in literature
		    \item {\color{blue} Current Undertaking: Design brain-inspired algorithms to handle PDE problems, especially PDEs of complex geometries and dynamic systems; To achieve this, I'm designing novel graph neural networks to adopt neural ansatz on unstructured mesh, following early work on graph-based simulation by deepmind etc.}
      \end{cvitems}
    }
% %---------------------------------------------------------
    \cventrynew
    {Variational Autoencoder Design and Implementation for Image-Driven Intelligent Structural Design} % Job title
    {06/2021-06/2022} % Date (s)
    {\textnormal{\textit{Supervised by Prof. Peng Hao of Department of Engineering Mechanics of DLUT}
    }}
    {
      \begin{cvitems} % Description(s) of tasks/responsibilities
        % \item {Developed and deployed {\bf RL} pipeline for robotic systems, integrating {\bf physics engines} ({\bf ROS}) to solve real-world problems.} 
        % \item {Drove advanced solutions, optimizing robotic platform performance under uncertainty through {\bf AI-driven systems}.}
        \item Conducted literature search and implement various variations VAEs, including vanilla VAEs, VAE with convolution architectures, VAE with ResNet shortcuts, etc.; trained them on the stiffener unit cell database, and record related numerical experimental results
		    \item Implemented the idea of VAE-GAN, using an additional discriminator loss to improve the quality of stiffener unit cell designs generated by neural networks; organized numerical results and plot illustrations of neural architectures
		    \item My VAE-GAN implementation was adopted in an intelligent optimization framework for grid-stiffened structures. In three typical numerical examples, compared with conventional stiffener unit cell designs, the obtained optimal designs were improved by 25.61\%, 25.88\%, and 10.66\%, respectively 
        \item Coauthored in the related published paper "Intelligent optimization of stiffener unit cell via variational autoencoder-based feature extraction" (See "Publications" section)
      \end{cvitems}
    } 
    \cventrynew
    {"Master of Structural Computation" APP Design} % Job title
    {01/2021-07/2021} % Date (s)
    {\textnormal{\textit{Supervised by Prof. Jun Yan of Department of Engineering Mechanics of DLUT}
    }}
    {
      \begin{cvitems} % Description(s) of tasks/responsibilities
        % \item {Developed and deployed {\bf RL} pipeline for robotic systems, integrating {\bf physics engines} ({\bf ROS}) to solve real-world problems.} 
        % \item {Drove advanced solutions, optimizing robotic platform performance under uncertainty through {\bf AI-driven systems}.}
        \item Studied the basics of structural mechanics, including displacement methods of bar system structures, stiffness matrix methods of bar system structures, direct stiffness methods, etc. 
		    \item Led "Master of Structural Computation" APP project. "Master of Structural Computation" can handle displacement calculation of bars and beams, and visualize nodal displacements on Android mobile devices
		    \item "Master of Structural Computation" APP has been granted a computer software copyright registration certificate
      \end{cvitems}
    } 
 %---------------------------------------------------------
    \cventrynew
    {The Seventh National Youth Science Popularization Innovation Experiment and Work Competition} % Job title
    {01/2021-05/2021} % Date (s)
    {\textnormal{\textit{Supervised by Prof. Dixiong Yang of Department of Engineering Mechanics of DLUT}
    }}
    {
      \begin{cvitems} % Description(s) of tasks/responsibilities
        % \item {Developed and deployed {\bf RL} pipeline for robotic systems, integrating {\bf physics engines} ({\bf ROS}) to solve real-world problems.} 
        % \item {Drove advanced solutions, optimizing robotic platform performance under uncertainty through {\bf AI-driven systems}.}
        \item Led the design and implementation of the "YiFen" APP, an Android-based solution for the real-time garbage image classification
        \item Utilized transfer learning techniques, employing the VGG backbone pre-trained on ImageNet; fine-tuned the fully-connected layers with garbage classification datasets to enhance model performance; deployed the trained deep learning model on mobile devices
        \item The fine-tuned model achieves 96.7\% accuracy on garbage classification datasets
        \item Won the $1^{\text{st}}$ Prize of Creative Work Unit - Intelligent Control Proposition (University Group)
      \end{cvitems}
    }
 %---------------------------------------------------------
    \cventrynew
    {2020 Liaoning Provincial Undergraduate Mathematical Modeling Competition} % Job title
    {11/2020-12/2020} % Date (s)
    {\textnormal{\textit{Supervised by Prof. Qiuhui Pan of School of Innovation and Entrepreneurship of DLUT}
    }}
    {
      \begin{cvitems} % Description(s) of tasks/responsibilities
        % \item {Developed and deployed {\bf RL} pipeline for robotic systems, integrating {\bf physics engines} ({\bf ROS}) to solve real-world problems.} 
        % \item {Drove advanced solutions, optimizing robotic platform performance under uncertainty through {\bf AI-driven systems}.}
        \item Developed a metacellular automata model to analyze the impact of the proportion of self-driving vehicles on traffic efficiency in single-lane, two-lane, and two-way four-lane traffic network models under different maximum road speed limits
        \item Established a NaSch model dividing the studied road into one-dimensional cells and related evolution rules to update the speeds and positions of the vehicles; Introduced stochastic slowing to represent the difference between automatic and non-automatic driving, the rule of lane-changing to stipulate the probability of lane-changing under specific circumstances
        \item Simulated specific highways with Python to obtain the spatio-temporal map, traffic density map and traffic efficiency-autonomous driving ratio map under the given parameters; concluded that the proportion of automated driving has a significant impact on the traffic efficiency under different maximum speeds and the two have a positive linear correlation
        \item Won the $1^{\text{st}}$ Prize of 2020 Liaoning Provincial Undergraduate Mathematical Modeling Competition
      \end{cvitems}
    }       
    \cventrynew
    {2020 National Undergraduate Mathematical Modeling Competition} % Job title
    {06/2020-09/2020} % Date (s)
    {\textnormal{\textit{Supervised by Prof. Qiuhui Pan of School of Innovation and Entrepreneurship of DLUT}
    }}
    {
      \begin{cvitems} % Description(s) of tasks/responsibilities
        % \item {Developed and deployed {\bf RL} pipeline for robotic systems, integrating {\bf physics engines} ({\bf ROS}) to solve real-world problems.} 
        % \item {Drove advanced solutions, optimizing robotic platform performance under uncertainty through {\bf AI-driven systems}.}
        \item Established a multi-objective control model for solder reflow oven temperature profile based on genetic algorithm  
        \item Analyzed the mathematical relationship between the temperature of the circuit board and the set temperature as well as the process speed of the solder reflow oven; established the circuit board temperature change model and the mechanism model of reasonable process speed; utilized the genetic algorithm to minimize the integral of the furnace temperature profile with respect to time above the hazardous temperature; developed a multi-objective planning model to achieve the optimal temperature profile
        \item Won the $2^{\text{nd}}$ Prize of 2020 National Undergraduate Mathematical Modeling Competition (Liaoning Region)
      \end{cvitems}
    }          
% %---------------------------------------------------------
%   \cventrynew
%     {NICE Lab \textnormal{-- Stretch RE1, Arizona, USA | \textit{Research Volunteer} (Prof. Zhe Xu) }} % Organisation
%     % {} % Job title % Job title
%     % {TamilNadu, India} % Location
%     {Jan 2023 - May 2024} % Date(s)
%     {
%       \begin{cvitems} % Description(s) of tasks/responsibilities
%         \item {Develop and evaluate differential control synthesis algorithms for multi-agent systems (Optimal Control \& MPC).} 
%         \item {Conduct perception and RL research with the Hello Robot, focusing on causal inference and counterfactuals for RL.}
%       \end{cvitems}
%       }


%---------------------------------------------------------
  % \cventrynew
  %   % {} % Organisation
  %   {Indian Institute of Technology Bombay, \textnormal{\textit{Robotic Software Engineer Intern} | Remote (India)}} % Job title
  %   % % {Bombay, India} % Location
  %   {May 2020 - Aug 2020} % Date(s)
  %   {
  %     \begin{cvitems} % Description(s) of tasks/responsibilities
  %       \item {Led an 8-person team to develop a fiducial-marker-based {\bf localization} model for an unstable camera feed.}
  %       \item {Optimized the localization model using V-rep for real-time camera feeds, achieved a calibration error of $\leq$  0.5\%.}
  %       \item {Orchestrated design, combined rule-based script and unit tested to validate auto-evaluators with 95\% coverage}
  %       % \item {Designed a rule-based visual scripting framework for configuring auto-evaluators through B0RemoteAPI.}
  %       % \item {Incorporated a {\bf unit testing} framework with automated test cases to validate the auto-evaluator model.}
  %     \end{cvitems}
  %   }
%---------------------------------------------------------
  % \cventrynew
  %   { e-Yantra, \textnormal{\textit{Robotic Engineer (Co-Founder and Team Lead)} | India  }} % Organisation
  %   % {} % Job title % Job title
  %   % {TamilNadu, India} % Location
  %   {Aug 2019 - July 2020} % Date(s)
  %   {
  %     \begin{cvitems} % Description(s) of tasks/responsibilities
  %       \item {Coordinated a 4-member team to National Finalist Status (Top 0.3\%), built a multi-tasking robot from scratch.} 
  %       % \item {Built a robot from scratch possessing vision, picking, placing, and autonomous decision-making (A* | Dijkstra) capabilities.}
  %       \item {Optimized {\bf pathfinding}(A* \& Dijkstra) algo. \& actions, reduced execution time by 22\%, enabled faster navigation.}
  %       \item {Integrated IR, proximity sensors for perception \& encoder motors, Servos for autonomous actions | Used {\bf CNC}.}
  %     \end{cvitems}
  %   }
%-----------------------------------------------------------------

  % \cventrynew
  %   { ABU Robocon 2020, \textnormal{\textit{Robotic Engineer (Team Member)} | India  }} % Organisation
  %   % {} % Job title % Job title
  %   % {TamilNadu, India} % Location
  %   {Jan 2019 - Feb 2020} % Date(s)
  %   {
  %     \begin{cvitems} % Description(s) of tasks/responsibilities
  %       \item {Directed team efforts, achieved top 15 (national) in ABU Robocon Stage 1 through innovative robot design.}
  %       \item {Engineered a 3-wheeled omni-drive system, achieved precise movement in any direction for the Pass Robot.} 
  %       \item { Modelled throwing hand compliance with pneumatics parameters using the Catapult mechanism and deployed.}
  %     \end{cvitems}
  %   }

% %---------------------------------------------------------
%   \cventrynew
%     {NLCIL \textnormal{(Neyveli Lignite Corporation India Ltd.) : \textit{Mechanical Engineer Intern} | Neyveli, India}} % Organisation
%     % {} % Job title % Job title
%     % {TamilNadu, India} % Location
%     {June 2019 - Aug 2019} % Date(s)
%     {
%       \begin{cvitems} % Description(s) of tasks/responsibilities
%         \item {Improved circulating water pump efficiency by 8.44\% using coating technologies with a payback period of < 2 months.} 
%         \item {Led team to assess and maintain water circulating pump through coatings, showcasing strong team leadership skills.}
%       \end{cvitems}
%       }

%---------------------------------------------------------
\end{cventries2}
